%
%   BACKGROUND
%
%   Introduce external and additional knowledge (the methods we did not learn)
%   Le Hieu did present some weird stuff in this section (also called Theory Basis)
\clearpage
\section{Background}
\subsection{Regression Model}
\subsubsection{Linear Regression Model}

\noindent 

\textbf{Multiple linear regression (MLR)}, also known simply as multiple regression, is a statistical technique that uses several explanatory variables to predict the outcome of a response variable. The goal of multiple linear regression is to model the linear relationship between the explanatory (independent) variables and response (dependent) variables. The formula of multiple linear regression:
$$
{y}_{i}={\beta}_{0}+{\beta}_{1} {x}_{i1}+{\beta}_{2} {x}_{i2}+...+{\beta}_{p} {x}_{ip}+\epsilon
$$
\textit{where:}
\begin{itemize}
    \item ${y}_{i}$ is the dependent variable.
    \item ${x}_{ip}$ is the explanatory variables.
    \item ${\beta}_{0}$ is the y-intercept (constant term).
    \item ${\beta}_{p}$ is the slope coefficients for each explanatory variable.
    \item $\epsilon$ is the model's error term (also known as the residuals).
\end{itemize}

When using linear regression, there are several assumptions that are typically made.

\textbf{Assumption 1: Linearity}, the relationship between the dependent variable and the independent variable is linear.

\textbf{Assumption 2: Independence}, the observations are independent of each other
$$
Cov({\varepsilon}_{i},{\varepsilon}_{j})=0,i\neq j
$$
\textit{where} $Cov({\varepsilon}_{i},{\varepsilon}_{j})$ is the covariance between the errors for observations $i$ and $j$.

\textbf{Assumption 3: Homoscedasticity}, the variance of the errors is constant across all levels of the independent variable(s)
$$
Var({\varepsilon}_{i})=\sigma ^{2},\forall i
$$
\textit{where} $Var({\varepsilon}_{i})$ is the variance of the error for observations $i$ and $\sigma ^{2}$ is a constant.

\textbf{Assumption 4: Normality}, the errors are normally distributed
$$
\varepsilon ~ \sim N(0,\sigma ^{2})
$$
\textit{where} $\varepsilon$ is the error term and $N(0,\sigma ^{2})$ denotes a normal distribution with mean 0 and variance $\sigma ^{2}$.
\subsubsection{Random Forest regression}
\textbf{Random Forest regression} is a supervised learning algorithm that uses ensemble learning method for regression. Ensemble learning method is a technique that combines predictions from multiple machine learning algorithms to make a more accurate prediction than a single model.

Random forest is an ensemble of decision trees. This is to say that many trees, constructed in a certain “random” way form a Random Forest, whether:
\begin{itemize}
    \item Each tree is created from a different sample of rows and at each node, a different sample of features is selected for splitting.
    \item Each of the trees makes its own individual prediction.
    \item These predictions are then averaged to produce a single result.
    \item The averaging makes a Random Forest better than a single Decision Tree hence improves its accuracy and reduces overfitting. A prediction from the Random Forest Regressor is an average of the predictions produced by the trees in the forest.
\end{itemize}

When using random forest regression, there are several assumptions that are typically made.

\textbf{Assumption 1: Independence of observations},  this assumption states that the observations in the dataset used for building the Random Forest regression model should be independent.

\textbf{Assumption 2: Input feature representation}, this assumption emphasizes the importance of appropriate representation of input features (independent variables).

\textbf{Assumption 3: Decision tree assumptions}, Random Forest regression is an ensemble of decision trees, and the assumptions of individual decision trees within the Random Forest ensemble apply.

\textbf{Assumption 4: Appropriate hyperparameter tuning}, Random Forest regression has several hyperparameters, such as the number of trees, the maximum depth of trees, the minimum number of samples required to split a node, among others. However, there is no specific mathematical formula for hyperparameter tuning, as it depends on the specific dataset and problem.
\subsubsection{Logistic regression}
\textbf{Logistic regression} is one of the most commonly used forms of nonlinear regression. It is used to estimate the probability of an event based on one or more independent variables. Logistic regression identifies the relationships between the enumerated variables and independent variables using the probability theory.

 A variable is said to be enumerated if it can possess only one value from a given set of values. Logistic Regression Models are generally used in cases when the rate of growth does not remain constant over a period of time. For example -when a new technology is introduced in the market, firstly its demand increases at a faster rate but then gradually slows down.
 
 When using Logistic regression, there are several assumptions that are typically made.
 
\textbf{Assumption 1: Binary outcome}, Logistic regression models the probability (P) of an event occurring, which is typically represented by a binary outcome variable (e.g., success/failure, yes/no, 1/0). The binary outcome variable can be denoted as Y, where Y = 1 represents the occurrence of the event of interest and Y = 0 represents the absence of the event.

\textbf{Assumption 2: Independence of observations}, Observations should be independent of each other, meaning that the value of the dependent variable for one observation should not be influenced by the value of the dependent variable for another observation.

\textbf{Assumption 3: Linearity of independent variables and log-odds}, Logistic regression models the log-odds (or logit) of the binary outcome, denoted as $ln(p/(1-p))$, where p is the probability of the event occurring. the relationship between the predictor variables (denoted as X) and the log-odds of the binary outcome is assumed to be linear. Mathematically, this can be represented as:
$$
ln\left(\frac{p}{1-p}\right)={\beta}_{0}+{\beta}_{1} {x}_{1}+{\beta}_{2} {x}_{2}+...+{\beta}_{k} {x}_{k}
$$
\textit{where:} ${\beta}_{0}$, ${\beta}_{1}$, ${\beta}_{2}$, ..., ${\beta}_{k}$ are the model parameters (coefficients) to be estimated, and ${x}_{1}$, ${x}_{2}$, ..., ${x}_{k}$ are the predictor variables.

\textbf{Assumption 4: Large sample size}, Logistic regression assumes a sufficiently large sample size to ensure reliable estimates of the model parameters. While there is no strict cutoff for sample size, a rule of thumb is to have at least 10-20 observations per predictor variable to ensure stable parameter estimates and reliable statistical inference.
\subsection{Statistics measurements}
\subsubsection{The Q-Q plot}
\noindent 

\textbf{The Q-Q plot}, or quantile-quantile plot, is a graphical tool to help us assess if a set of data plausibly came from some theoretical distribution such as a Normal or exponential. For example, if we run a statistical analysis that assumes our residuals are normally distributed, we can use a Normal Q-Q plot to check that assumption. It’s just a visual check, not an air-tight proof, so it is somewhat subjective. But it allows us to see at-a-glance if our assumption is plausible, and if not, how the assumption is violated and what data points contribute to the violation.

A Q-Q plot is a scatterplot created by plotting two sets of quantiles against one another. If both sets of quantiles came from the same distribution, we should see the points forming a line that’s roughly straight.
\subsubsection{R-Squared $(R^{2})$}
\noindent 

$R^2$ is a statistical measure that represents the proportion of the variance for a dependent variable that’s explained by an independent variable in a regression model.

Whereas correlation explains the strength of the relationship between an independent and a dependent variable, R-squared explains the extent to which the variance of one variable explains the variance of the second variable. So, if the $R^{2}$ of a model is 0.50, then approximately half of the observed variation can be explained by the model’s inputs.

The formula for R-squared:
$$
{R}^{2} = 1 - \frac{\textup{Unexplained Variation}}{\textup{Total Variation}}
$$

The calculation of R-squared requires several steps. This includes taking the data points (observations) of dependent and independent variables and finding the line of best fit, often from a regression model. From there, you would calculate predicted values, subtract actual values, and square the results. This yields a list of errors squared, which is then summed and equals the unexplained variance.

The \textbf{adjusted coefficient of determination} is the multiple coefficient of determination $R^2$ modified to account for the number of variables and the sample size. It is calculated by
$$
\textup{Adjusted}\enskip R^2=1-\dfrac{n-1}{n-(k+1)}\times(1-R^2)
$$
\subsubsection{P-Value}
\noindent 

In statistics, the p-value is a measure of the evidence against a null hypothesis. It is the probability of observing a test statistic as extreme as, or more extreme than, the one calculated from the data, assuming that the null hypothesis is true.

In other words, the p-value is the probability of obtaining the observed results or more extreme results, assuming that the null hypothesis is true. If the p-value is low (usually less than 0.05), it suggests that the observed results are unlikely to be due to chance and provides evidence against the null hypothesis. Conversely, if the p-value is high, it suggests that the observed results are likely to be due to chance, and there is insufficient evidence to reject the null hypothesis.

The p-value is an important concept in hypothesis testing, which is a common statistical method used to make decisions based on data. It helps researchers determine whether the observed data supports or contradicts a particular hypothesis.
\subsection{Analysis of Variance ANOVA}

Analysis of variance (ANOVA) is a statistical method used to test for differences among two or more population means by analyzing the variances of samples taken from the populations.

One-way ANOVA is a statistical method to compare the variances of multiple levels of a single factor.

For each observation under the treatment $i$ under the $j$ observation called $y_{ij}$ we have the linear combination:

\[y_{ij} = \mu + \tau_i + \epsilon_{ij} 
\begin{cases}
    i = 1,2,...,a.\\
    j = 1,2,...,n.
\end{cases}
\]
\textit{where,}
\begin{itemize}
    \item $\mu$ is the overall mean.
    \item $\tau_i$ is the effect of the $i$th treatment effect.
    \item $\epsilon_{ij}$ is a random component error.
\end{itemize}

We could rewritten the model as.

\[y_{ij} = \mu_i + \epsilon_{ij} 
\begin{cases}
    i = 1,2,...,a.\\
    j = 1,2,...,n.
\end{cases}
\]
\textit{where,}
\begin{itemize}
    \item $\mu_i$ =  $\mu + \tau_i$
    \item $\tau_i$ is the effect of the $i$th treatment effect.
    \item $\epsilon_{ij}$ is a random component error.
\end{itemize}

To perform ANOVA, the following assumption is made: $\epsilon_{ij}$ is normally and independently 
distributed : $\epsilon_{ij} \approx N(0, \sigma^2)$, and each treatment is a sample that follows $N(0, \sigma^2)$.


\begin{enumerate}
    \item Normality: The populations have distributions that are approximately normal.
    \item Homoscedasticity  : The populations have the same variance
    \item Independent: the data is random and independent.
\end{enumerate}

However the Normality and Homogeneity of variance are only loose requirement as the method still well despite failing these assumption\cite{mont03}
However we will also use the Kruskal - Wallis test for anything that do not sastify the assumption.
We want to test the Null hypothesis:
\[
\begin{cases}
    H_0: \mu_1 = \mu_2 = ... = \mu_n \\
    H_1: \text{two mean are different}
\end{cases}
\]

Total sum of squares:
\[ SS_T = \sum_{i = 1}^{a} \sum_{j = 1}^{n} (y_{ij} - \bar{y})^2\]
\[ SS_T = n\sum_{i=1}^{a}(\bar{y_i}-\bar{y})^2 + \sum_{i=1}^{a}\sum_{j=1}^{n}(y_{ij}-\bar{y_i})^2\]

or
\[ SS_T = SS_{Treatment}+SS_{Error}\]

where degree of freedom is:
\[df(SS_T) = N - 1 \quad df(SS_{Treatment}) = a - 1 \quad df(SS_{Error}) = N - a \]

Mean square for treatments: 
\[MS_{Treatment} = SS_{Treatment} / df(SS_{Treatment})\]
\[MS_{Treatment} = SS_{Treatment} / (a - 1)\]

Mean square for error: 
\[MS_{Error} = SS_{Treatment} / df(SS_{Error})\]
\[MS_{Error} = SS_{Treatment} / (N - a)\]   

F test statistic: 
\[F_0 = \frac{MS_{Treatment}}{MS_{Error}}\]

If \[F_0 > F_{\alpha , a-1,a(n-1)}\]

\subsection{Kruskal - Wallis H-Test}
Kruskal - Wallis test which uses ranks of data from three or more independent simple random samples to test the null hypothesis that the samples come from populations with the same median.
The Kruskal-Wallis test for equal medians does not require normal distributions, so it is a distribution-free or non-parametric test. 
In applying the Kruskal - Wallis test we need to compute the test statistic H.
\[H = \frac{12}{N(N+1)*\sum_{i=1}^{k}\frac{R_i^2}{n_i}-3(N+1)}\]
\textit{where:}

\begin{itemize}
    \item $N$ is the number of values from all combined samples.
    \item $R_i$ is the sum of ranks from a particular sample, and $n_i$ is the number of values from the corresponding rank sum.
    \item $n_i$ is the number of values from the corresponding rank sum.
\end{itemize}

\begin{cases}
    H_0: \text{The samples come from populations with the same median. }\\ 
    H_1: \text{The samples come from populations with medians that are not all equal.}
\end{cases}
\subsection{Levene test}
Levene's test is used to test if k samples have equal variance. In this assignment, we will use it as the primary tool for testing the Homogeneity of variance.

Given a variable Y with sample of size N divided into k subgroups where $N_i$ is the sample size of the $i$th subgroup, the Levene test is defined as:
\[
\begin{cases}
    H_0: \sigma_1^2 = \sigma_2^2 =...=\sigma_k^2 \\ 
    H_1: \text{there are at least one pair with unequal variance.}
\end{cases}
\]
\[W = \frac{(N-K)}{(k-1)}\frac{\sum_{i=1}^{k}N_i(\bar{Z_i}-\bar{Z})^2}{\sum_{i=1}^{j}\sum_{j=1}^{N_i}(Z_{ij}-\bar{Z_i})^2}\]
\textit{where} $Z_{ij}$ cahn have one of these following definitions:

\begin{itemize}
    \item $Z_{ij} = Y_{iJ} - \bar{Y_i}$ where $\bar{Y_i}$ is the mean  of the $i$th subgroup.
    \item $Z_{ij} = Y_{iJ} - \tilde{Y_i}$ where $\tilde{Y_i}$ is the median of the $i$th subgroup.
    \item $Z_{ij} = Y_{iJ} - \bar{Y_i}^{'}$ where $\bar{Y_i}^{'}$ is the trimmed mean of the $i$th subgroup.
\end{itemize}

The three choice for detemining $Z_{ij}$ determine the robustness and power of Levene's test. We will choose choice where $\tilde{Y_i}$ is the median as it is the default choice of LeveneTest in R

\subsection{Shapiro-Wilk test}
The Shapiro-Wilk test, calculates a W statistic that tests whether a random sample, $x_1$, $x_2$, ..., $x_n$ come from a normal distribution. 
The W statistic is calculated as:
\[W = \frac{(\sum_{i=1}^{n}a_ix_{(i)})^2}{\sum_{i=1}^{n}(x_i-\bar{x})^2}\]
\textit{where:}
\begin{itemize}
    \item $x_{(i)}$ are the ordered sample values
    \item $a_i$ are the constant generated from the means, variance and covariance of the order of a sample of size n from a normal distribution.
    \item $\bar{x}$ is the sample mean
\end{itemize}

We would like to use this test to test the Null hypothesis:
\[
\begin{cases}
    H_0: \text{the population is normally distributed} \\
    H_1: \text{the population is not normally distributed}
\end{cases}
\]
if the p-value is less than $\alpha$ then we can reject the null hypothesis this test and consider our data to not be Normally distributed.

\subsection{Post-hoc comparison tests}

Post-hoc comparison tests are usually used to identify the differences between multiple groups after the study has been concluded.
Specifically, it is used to see the pairwise differences between different groups of the dataset after an ANOVA is done and the conclusion
has been drawn that there is statistically significant result.\cite{foster22}

\subsubsection{Tukey HSD test}

Tukey HSD's test compares the means of every treatment to the means of every other treatment; 
that is, it applies simultaneously to the set of all pairwise comparisons

Tukey HSD test perform a pairwise comparison between the means of the treatments by seeing whether their difference
is statistically significant as compared to the expected standard error. It makes use of studentized range statistic:

\begin{equation}
    Q = \frac{\bar{y}_{\text{max}} - \bar{y}_{\text{min}}}{SE}
\end{equation}
\textit{where,} $\bar{y}_{\text{max}}$ and $\bar{y}_{\text{min}}$
are the largest and smallest sample means, respectively.

This test indicates two means are different if $Q > g(\alpha,f)\times S$\cite{tukey},
where, $S$ is the standard error of this statistic, and $g(\alpha,f)$ is studentized range 
distribution of significant level $\alpha$ and even degree of freedom $f$.

\subsubsection{Dunn's z-test}

Dunn's z-test statistic approximates exact rank-sum test statistics by using the
mean rankings of the outcome in each group from the preceding Kruskal-Wallis test
and basing inference on the differences in mean ranks in each group.\cite{dinno15}

The statistic to compare the difference in mean between group A and group B.
\begin{equation}
    z = \frac{\bar{W}_A - \bar{W}_B}{S}
\end{equation}
\textit{where} $S$ is the standard error.

\begin{equation}
    S = \sqrt{(\frac{N(N+1)}{12} - \frac{\sum^{s=1}_{r} \tau^3_{s} - \tau_s}{12(N-1)})(\frac{1}{n_A} + \frac{1}{n_B})}
\end{equation}
\textit{where,} $N$ is the total number of observations across all groups, $r$ is the number of tied
ranks, and $\tau_s$ is the number of observations tied at the $s$-th specific tied value.

In our project, we use The Bonferroni's $p$-value adjustment, which basically multiply the $p$-value with a constant $m$.



